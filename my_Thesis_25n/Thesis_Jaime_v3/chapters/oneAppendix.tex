\chapter{Example RDMA application}

%\section{Detailed Validation Results}

\label{chapter:appendixA}



\begin{lstlisting}[label=listing:appendixA, caption={}, morekeywords={clrdma_init_altera, clrdma_init_nvidia, clrdma_create_pinnable_buffer_nvidia, clrdma_get_buffer_address_altera, clrdma_get_buffer_address_nvidia, read_rdma, cl_rdma}]
#include <iostream>
#include <CL/opencl.h>
#include "cl_rdma.c"

using namespace std;

//transfer size 32 MB
#define N (1024 * 1024 * 32)
#define ALIGN_TO 4096

int main(int argc, char* argv[])
{
cl_uint n_platforms;
cl_context       contexts[n_platforms];
cl_device_id     devices[n_platforms];
cl_platform_id   platformID[n_platforms];
cl_mem           buffers[n_platforms];
cl_command_queue queues[n_platforms];
int              altera_idx=-1;
int              nvidia_idx=-1;

clGetPlatformIDs(0, NULL, &n_platforms);
if(n_platforms < 2)
{ cout<<"Found only "<<n_platforms<<" OpenCL platforms"<<endl; return(1); }

clGetPlatformIDs(n_platforms, platformID, NULL);
for(unsigned i = 0; i<n_platforms; i++)
{
	char chbuffer[1024];
	clGetPlatformInfo(platformID[i],CL_PLATFORM_NAME,1024,chbuffer,NULL);
	if(strstr(chbuffer,"NVIDIA") != NULL){nvidia_idx = i;}
	if(strstr(chbuffer,"Altera") != NULL){altera_idx = i;}
	
	clGetDeviceIDs(platformID[i],CL_DEVICE_TYPE_DEFAULT,1,&devices[i],NULL);
	clGetDeviceInfo(devices[i],CL_DEVICE_NAME,sizeof(chbuffer),chbuffer,NULL);
	contexts[i]=clCreateContext(0,1,&devices[i],NULL,NULL,NULL);
	queues[i]=clCreateCommandQueue(contexts[i],devices[i],0,NULL);
}
if(nvidia_idx==-1)
{cout<<"NVIDIA platform not available!"<<endl;return(1);}
if(altera_idx==-1)
{cout<<"Altera platform not available!"<<endl;return(1);}

//initialize RDMA
int rc=clrdma_init_altera(contexts[altera_idx], devices[altera_idx]);
if     (rc==1){cout<<"Altera Driver is not loaded!"<<endl; return(1);}
else if(rc==2){cout<<"Wrong Altera Driver is loaded!"<<endl; return(1);}
else if(rc==4){cout<<"Could not load get_address.aocx!"<<endl; return(1);}
else if(rc){cout<<"Failed to initialize rdma for Altera."<<endl; return(1);}

rc=clrdma_init_nvidia(contexts[nvidia_idx]);
if     (rc==1){cout<<"Nvidia Driver is not loaded!"<<endl; return(1);}
else if(rc){cout<<"Failed to initialize rdma for NVIDIA."<<endl;return(1);}

rc=clrdma_create_pinnable_buffer_nvidia(contexts[nvidia_idx], 
                                                      queues[nvidia_idx], 
                                                     &buffers[nvidia_idx], N);
if(rc){cout<<"Failed to create a pinnable buffer!"<<endl; return(1);}
buffers[altera_idx]
  =clCreateBuffer(contexts[altera_idx],CL_MEM_READ_WRITE,N,NULL,NULL);

unsigned long addresses[2];
rc=clrdma_get_buffer_address_altera(buffers[altera_idx], 
                                                queues[altera_idx], 
                                               &addresses[altera_idx]);
if(rc){cout<<"Could not get address for FPGA buffer."<<endl;return(1);}
rc=clrdma_get_buffer_address_nvidia(buffers[nvidia_idx], 
                                                queues[nvidia_idx], 
                                               &addresses[nvidia_idx]);
if(rc){cout<<"Could not get address for GPU  buffer."<<endl;return(1);}

unsigned char* data0; unsigned char* data1;
if(posix_memalign((void**)&data0,ALIGN_TO,N))
{	cout<<"Could not allocate aligned memory!"<<endl; return(1);	}
if(posix_memalign((void**)&data1,ALIGN_TO,N))
{	cout<<"Could not allocate aligned memory!"<<endl; return(1);	}
for(unsigned i=0;i<N;i++){data0[i] = i%256; }
clEnqueueWriteBuffer(queues[altera_idx],buffers[altera_idx],
                            CL_TRUE,0,N,data0,0,NULL,NULL);

//RDMA transfer from FPGA to GPU of size N (timeout after 5 seconds)
rc = read_rdma(addresses[altera_idx], addresses[nvidia_idx], N, 5);
if(rc){cout<<"RDMA transfer failed. Code:"<<rc<<endl;return(1);}

clEnqueueReadBuffer (queues[1],buffers[1],CL_TRUE,0,N,data1,0,NULL,NULL);
//check whether the data is correct
for(unsigned i=0;i<N;i++)
{
	if(data0[i]!=data1[i])
	{cout<<"Transferred data is incorrect!"<<endl;break;}
}

free(data0); free(data1);
return(0);
}





\end{lstlisting}


