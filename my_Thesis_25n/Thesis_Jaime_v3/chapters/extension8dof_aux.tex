\chapter{Extension to the 8 DOF robot}
\label{ch:extension8dof}





\section{Gradient-based minimization}
\label{sec:Gradientbasedminimization}

This method consists in project the gradient of the reflected mass into the null space. Chosing $k$ to be a positive scalar in (\ref{eq:gbm_nico}) a null space torque that minimizes $m_u$ is ensured. 
\textcolor{red}{Explain here as well the stuff from Nemec and how the gradient has to be premultiplied by $M^{-1}$...but like it is in torque level I have to do the demonstration.} As in  \cite{paper_iros2017} the extension here done to the 8-DOF robot is for an unconstrained optimization problem, i.e. only the partial derivatives of the reflected mass w.r.t. the joints are considered. 
In (\ref{eq:gbm_nico}) the gradient of the reflected mass was projected into the null space using the well known projector from \cite{khatib1995}. The same projector is used here in order to ensure static and dynamic consistency.  The equation (\ref{eq:reflected_robot_mass}) is used to expand the gradient as

\begin{equation}
\nabla m_u(\mathbf{q}) = 
\frac{\partial {m_u(\mathbf{q})}}{\partial {q_i}} = \frac{\partial {[\mathbf{u^T} \Lambda_{v}^{-1}(\mathbf{q}) \mathbf{u}]^{-1}}}{\partial {q_i}}, \label{eq:grad_refl_mass_1}
\end{equation}
where, in our case with the 8-DOF,  $i = 1, \dots, 8$. This equation can be rewritten as
\begin{equation}
\nabla m_u(\mathbf{q}) = - \left [ \mathbf{u^T} \frac{\partial {\Lambda_{v}^{-1}}}{\partial {q_i}} \mathbf{u} +
\frac{\partial {(\mathbf{u^T} \Lambda_{v}^{-1} \mathbf{u})}}{\mathbf{u}} \frac{\partial {\mathbf{u}}}{\partial {q_i}} \right ] m_u^2 \label{eq:grad_refl_mass_2}
\end{equation}




 As with the 7-DOF LWR, the 8-DOF robot shows bad performance close to local maxima. This is an expected result because the mass matrix, and the Jacobian, do not depend on the linear axis position (see (\ref{eq:j_m_no__q1})). The rotational joints are not able to overcome the friction even for high values of $k$. This problem is not present in the linear axis which has a position controller. The linear axis is included into the whole-body impedance control framework using an admittance interface \cite{whole_body_imp}. Therefore, even small values for the torque cause a displacement of this joint. However, to fully exploit the redundancy of the robot  in the optimization, making use of the elbow, another approach is needed. 
 
 \textcolor{red}{I HAVE TO TEST THIS ON THE RT MACHINE AND SEE THAT THERE ARE LOW TORQUES}
 
 
\textcolor{red}{ SINCE THE COMPUTATION OF THE PSEUDOINVERSE IS EXPENSIVE MAYBE IT COULD BE DONE THE IMPOSITION OF THE CONSTRAINT FOR THE ACCELERATIONS. AS IT IS DONE LATER TO OBTAIN A NULL SPACE VELOCITY, BUT HERE USED TO OBTAIN A NULL SPACE TORQUE}

%% HERE THE SADDLE POINTS DO NOT MAKE SENSE ANYMORE. And not only close to maxima, but close to saddle point, which makes sense as the null  problem is characteristic of two dimensional data. 



%One could think of obtaining new equations of motion for the robot, where the constraint of null space movement would be satisfied. Which using the Lagragian approach would be getting from 
%
%\begin{equation}
%M(\mathbf{q})\ddot{\mathbf{q}}+C(\mathbf{q},\dot{\mathbf{q}})+G(\mathbf{q})=\mathbf{\tau} +J_c*\mathbf{\Lambda},
%\label{joint_space_dynamics}
%\end{equation}	
%to	
%
%\begin{equation}
%\bar{M}(\mathbf{q})\ddot{\mathbf{q}}+\bar{C}(\mathbf{q},\dot{\mathbf{q}})+\bar{G}(\mathbf{q})=\bar{\mathbf{\tau}},
%\label{new_joint_space_dynamics}
%\end{equation}	
%
%where $J_c$ is the [6x8] constrained Jacobian that express the null space movement and $\Lambda$ are the lagrangian multipliers. But one can see that none of the matrices in the new equation of motion would depend on $q_1$ as none of the originals depends on $q_1$, so this would actually solve nothing.














\section{Minimization based on attractive potential}
\label{sec:Minimattractivepotential}




For the potential field  defined in (\ref{eq:potential_intro}) the aim is to find a suitable goal position, where the reflected mass on the tip of the robot in certain direction is minimal. Two possible goal positions are: Global minimum, determining all possible null space positions and then finding the local minima and global minimum. This is treated in  Chapter \ref{ch:globaloptimization}. And local minimum: an iterative solution is discussed in  Chapter \ref{ch:localoptimization}.
Since the goal position is chosen to be far from local maxima the problem from the previous approach does not appear here.

%\begin{equation}
%U(\mathbf{q}) = - \frac{1}{2} Kp (\mathbf{q}_{m_u}^\ast - \mathbf{q})^T (\mathbf{q}_{m_u}^\ast - \mathbf{q}). \label{eq:potential_extension8}
%\end{equation}

%Differentiating we obtain a torque to project into the null space
%
%\begin{equation}
%\mathbf{\tau}_{m_u}^\ast = \frac{\partial {U(\mathbf{q})}}{\partial {\mathbf{q}}} = - k (\mathbf{q}_{m_u}^\ast - \mathbf{q}).
%\end{equation}
%
%
%\begin{equation}
%\mathbf{\tau}_d = \mathbf{\tau}_\mathrm{imp} + (I - J^T J^{\#T}) \mathbf{\tau}_{m_u}^\ast, \label{eq:potential_controller_extension8}
%\end{equation}





%\section{Conclusion}
%\label{sec:conclusion_ext8}







%
%\subsection{Null space projection ???}
%\label{subsec:nsprojection}
%
%\textsc{{\color{red}} comparar diferentes projecciones?, hablar de el mayor error que obtengo al usar el projector que sin él para determinadas u??}
%
%{{\color{red}} THIS IS FROM NICO icra2016 paper: 
%	We project this torque into the null space
%	\begin{equation}
%	\mathbf{\tau}_d = \mathbf{\tau}_\mathrm{imp} + (I - J^T J^{\#T}) \mathbf{\tau_{m_u}}^\ast, 
%	\end{equation}
%	where we select the mass-weighted pseudoinverse\footnote{The mass-weighted null space projection is given by \mbox{$N = (I - J^T (J M^{-1} J^T)^{-1} J M^{-1})$}.} \cite{khatib1995} to achieve dynamic consistency. In the next section we conduct experiments for static and dynamic end effector positions to demonstrate the performance of the proposed controller.}
%
%
%
%








