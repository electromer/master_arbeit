\chapter{Conclusion and Outlook}
\label{ch:conclusion}


This thesis deals with the problem of minimizing the reflected mass for a 7-DOF LWR mounted on a linear axis. Needing a null space scheme for this minimization, the first approaches considered are in torque level. Projecting the gradient of the reflected mass into the null space shows poor performance close to maxima, where the commanded torques are not able to overcome the joints friction. An attractive potential approach overcomes this problem. For this potential, first the goal minimum is considered as goal position. This approach may be done offline, for a previously known trajectory of the robot. Offline computation would also include path planning to the minimum. This is necessary because the robot may go through local peaks of maximum reflected mass, on his trajectory to the global minimum. However, most of the tested grids show contour lines where the global minimum lies. This causes high discontinuity making this goal position undesired. 

For the search of the local minimum two fundamental optimization techniques are considered: Trust region and line search. The trust region techniques require an analytical form of the reflected mass constrained to the null space motion. An attempt is done using curve-fitting, but not being successful, the line search techniques are preferred. For this a null space velocity that minimizes the reflected mass is needed. The Projected Gradient and the Reduced Gradient arise as possible approaches to obtain this velocity. Since we are minimizing locally, real-time capabilities in the final controller are pursued. The Reduced Gradient is therefore chosen for being more computationally efficient. This method also allows decoupling on the minimization. And exploiting this characteristic a joint limit avoidance strategy is developed.  The Reduced Gradient is obtained analytically and numerically. Inherent to the gradient descent method, oscillations close to the minima appear during the minimization. This happens when the problem is ill-conditioned (high condition number). Decreasing the step size causes slow motion, so other line search techniques are implemented in order to decrease oscillations and increase convergence rate. Three Conjugate Gradient Descent methods are tested, showing in general worse convergence than the original Gradient Descent. Two Quasi-Newton methods are tested as well, showing in general not a significant improvement in the convergence rate w.r.t. the Gradient Descent. Therefore, the Gradient Descent is chosen for simplicity. Still showing oscillations, an algorithm is developed in which several iterations are computed in order to detect and decrease these oscillations. Finally a  non-normalized joint velocity is chosen. This bounded velocity,  shows good dynamics in the minimization and ensures no oscillations close to minima.

The implementation of the Gradient Descent algorithm is implemented in a real-time capable controller. The controller is tuned for both, the analytical and the numerical Reduced Gradient. The addition of damping is necessary to ensure good behavior of the joints, specially linear axis. The analytical approach shows to be more computationally efficient, and the numerical approach shows to have more robust parameters.
Finally the controller is implemented in parallel with the SMU. The SMU ensures safe motion of, while the controller here developed shows to improve the performance of the robot.


\textcolor{magenta}{\section{Treatment of the oscillations}
\label{sec:oscillations_conclusions}
%
\textcolor{red}{Influence of the nmax: But close to the minimum the torques are low because the algorithm does not require a high number of iterations to find the minimum. There are not permanent oscillations close to the minimum because the small torques cannot overcome the friction of the robot. }
%
An inherent problem of the line search methods is the appearance of oscillations in the presence of narrow valleis. This happens when the problem is ill-conditioned (high condition number). 
%
When testing the GD algorithm in the grid, this zigzag appeared. A first try is decrease the step size till there is no more oscillations. For a normalized velocity, the low step size slows down too much the convergence rate.
%
The second attempt is the use of CGD methods. However close to minima, the friction term is enlarged in excess, impoverishing the convergence w.r.t. the GD.
%
The theoretical better convergence rate of the QN may compensate the slow motion close to the minima. But, the often loss of positive definiteness, requires a restart of the iterative strategies. Causing the speed of these methods not to be significantly higher, that the one from the GD.
%
In the algorithm (see line \ref{2D_alg:line:zigzag}), the oscillations are finally decreased by setting a threshold for the convergence. Under which the position is ensured to be close to the minima.
%
To complement the previous threshold, and decrease even more the probability of oscillations in the real robot, another strategy is also used. This strategy is implemented in the line \ref{2D_alg:line:zigzag} of the Algorithm, once the algorithm has been implemented on the controller. The algorithm runs $n_{max}$ iterations, before commanding a goal position. If, while running, oscillations are detected, this parameter is increased.  Therefore the actually commanded positions will suffer from less oscillations. 
%
As mentioned in the Chapter \ref{ch:experiments}, if the number of iterations is too high the computational efficiency decreases significantly. Using a normalized velocity causes poor dynamics, even for high number of iterations.
As a result, velocities are chosen to be bounded instead. This improves the dynamics when far from the minimum, and decreases the oscillations close to the minima.
%
\textcolor{red}{\textbf{I REALLY HAVE TO CHECK IF THE GOOD DYNAMICS FAR FROM THE MINIMUM ARE CAUSE OF THE NON-NORMALIZED dq, BETTER THAN THE NORMALIZED ONE (*DT) BECAUSE IT DOES SEEM TO MAKE MUCH SENSE. THIS WOULD ONLY ENSURE LOWER VELOCITIES CLOSE THE MINIMUM, BUT NOT MUCH MORE}}
%
With the two last mentioned strategies, together with non normalized joint velocities, none oscillations appear in the real robot.
%
\section{Experiments}
results of comparing the analytical vs numerical in the controller and how at the end looks cool with SMU 
%
The experiments were first run on simulation and then on the real robot but the testing at high speeds was not possible in the real robot because collisions were acknowledged and even deactivating the collision-detection strategy the commands were lost, meaning that the interface does not let the robot to  move at such speeds and accelerations. At low speeds the experiments were successfully run in the real robot and these are the results presented.
%
There is a spring behavior to keep the TCP not too far from the linear axis but this has not been yet projected into the null space so these experiments are done without these virtual springs.
%
In Fig.\ref{fig:SmuExp_Vel_vmax05} and Fig.\ref{fig:SmuExp_Vel_vmax05} the velocities go slightly over the maximum allowed. This is due to the performance of the interpolator used.
%
The controller is tuned and a set parameters which lead to good behavior (stable\textcolor{red}{ delete "stable" till I can prove it} and good dynamics) is found. This is done for both the analytical and the numerical case. 
1}
\section{Future work}
\label{sec:future_work}

\subsection{Generalization to n-DOF}
\label{subsec:extension_ndof}

